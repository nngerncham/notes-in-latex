\chapter{Superposition and Standing Waves}

Similar to particles, we can combine multiple waves in a system
with boundary conditions. In this system, only certain frequencies
can exist and we say the frequencies are quantized. 

Also consider the combination of waves with different frequencies.
When two sound waves with similar frequency interfere, we notice 
variations in the loudness called \textit{beats}.

\section{Analysis Model: Waves in Interference}

Many waves in nature cannot be described by a single traveling wave, 
so we must analyze them as a combination of multiple waves. For this,
we can use the \textbf{superposition principle}: \textit{If two or
more traveling waves are moving through a medium, the resulting
wave is the value of the wave function at any point is the algebraic
sum of the values of the wave functions of the individual waves}.
Waves that follow this principle are called \textbf{linear waves}.

When two waves combine in the same region of space, it results in a new wave 
that is the combination of the two original waves. This phenomena is called
\textbf{inteference}. There are two main types of inteference:
\begin{itemize}
    \item \textbf{Constructive interference}: the amplitudes of both original waves are
        positive, so the resultant wave has even bigger amplitude
    \item \textbf{Destructive interference}: the amplitudes of both original waves have
        opposite signs, so the absolute resultant wave would be difference of absolutes
        of the original waves
\end{itemize}

\subsection{Superposition of Sinusoidal Waves}

Consider two waves sinusoidal waves with the same amplitude but with different phase.
We can express them as 
\[ y_1 = A\sin(kx - \omega t) \quad y_2 = A\sin(kx - \omega t + \phi) \]
If we combine the two waves, we will get
\[ y = y_1 + y_2 = A\left[\sin(kx - \omega t) + \sin(kx - \omega t + \phi)\right] \]

Using the trig-identity $\sin a + \sin b = 2\cos((a-b)/2)\sin((a+b)/2)$,
we can simplify the resultant wave to 
\begin{equation}
    y = 2A\cos\left(\frac{\phi}{2}\right)\sin\left(kx - \omega t + \frac{\phi}{2}\right)
\end{equation}
Here, notice that the amplitude of $y$ becomes $2A\cos(\phi/2)$ and the phase becomes $\phi/2$

Depending on the phase constant, the resultant wave $y$ will have one of the following amplitudes
\begin{itemize}
    \item $\phi = 0$: $\cos(\phi/2) = \cos(0) = 1\ \therefore\ A_y = 2A$ and the crest
        of $y$ are at the same location as $y_1$ and $y_2$
        (This also has a name $-$ \textit{in-phase})
    \item $\phi = \pi$ or any of its odd multiple: the crest of one wave is at the position of
        the trough of the other, causing a destructive interference leading to $A_y = 0$ since 
        the amplitudes of $y_1$ and $y_2$ are the same
    \item $\phi = 2\pi$ or any other even multiple: $A_y$ will be somewhere between $0$ and $2A$
\end{itemize}

\subsection{Interference of Sound Waves}

The distance along given path from source to receiver is called the \textbf{path length} $r$.
From the above diagram, notice that $r_1$ is fixed while $r_2$ can vary by shifting the grey tube.
The difference in path length can be computed by $\Delta r = |r_2 - r_1|$.
\begin{itemize}
    \item If $\Delta r = 0$ or is an integer multiple of the wavelength, the two waves will
        will reach the receiver in phase, creating a constructive inteference, also with max amplitude
    \item If $\Delta r = \lambda/2$ or some odd multiple of it, the two waves are exactly
        $\pi$ radian out of phase and will cancel each other out 
\end{itemize}

\section{Standing Waves}

Suppose we have two speakers facing each other playing the same sounds but in opposite direction,
we can express the wave function for both as 
\[ y_1 = A\sin(kx - \omega t)\quad y_2 = A\sin(kx + \omega t) \]
We can use trigonometric identity $\sin(a \pm b) = \sin a\cos b \pm \cos a \sin b$
to reduce their sum down to \begin{equation}\label{18.1}
    y = 2A\sin(kx)\cos(\omega t)
\end{equation}
This equation represents the wave function of a \textbf{standing wave}, meaning that it is
the wave does not seem to be traveling, just oscillating within the outline of $2A\sin(kx)$.

Notice that when $\sin(kx) = 0$, when $kx = n\pi$ for some integer $n$, or when $x =
(n\lambda)/2$ for some integer $n$ (since $k=\lambda/2$), the amplitude will always
be zero. These points are called the \textbf{nodes}, basically the point with no oscillation.

In contrast, elements of the string with the greatest possible displacement from equilibrium
with amplitude of $2A$ are called \textbf{antinodes}. These are located when $\sin(kx) = \pm 1$.
These occur when $x = n\lambda/4$ for some odd integer $n$.

\section{Standing Waves in Air Columns}

From the definition of nodes, we know that each node are $\lambda/2$ distance away from each
other. Since antinodes are in the middle between two nodes and vice versa, we can say that
the distance between a node and an antinode is $\lambda/4$.

Consider a first (fundamental) harmonic standing wave in a pipe of length $L$ where one end is
closed and is the point of a node, and the other end is open and is the point of an antinode,
this means that the length of the tube is related to the wavelength by 
\[ L = \lambda/4 \quad \lambda = 4L \]
We can define the wavelength and frequency as 
\[ \lambda_1 = 4L\quad f_1 = \frac{v}{\lambda_1} = \frac{v}{4L} \]

If we were to add more \textit{flips} or another pair of node and antinode into the pipe, we can 
model the wavelength and the frequency by the following 
\[ \lambda_i = \frac{4}{i}L\quad f_i = i f_1 \]
and it is called the $i$-th harmonic wave. Note that $i$ is an odd integer since $L = \lambda/4$ and 
adding another a node and an antinode into the pipe means that we need to fit another $\lambda/2$
in to $L$.

\cim{images/ch18/fig18_13b}{0.4}

In the case that the pipe with length $L$ has two open ends, both ends will need to be the antinode
with a node in the middle of each. Knowing that antinodes are $\lambda/2$ apart from each other and 
the antinodes at each end are $L$ apart, we can define the wavelength and frequency as
\[ \lambda_1 = 2L\quad f_1 = \frac{v}{\lambda_1} = \frac{v}{2L} \]

If we want to add more \textit{flips} or antinodes into the pipe, we can model the wavelength and
frequency by the following \[ \lambda_i = \frac{2}{i}L \quad f_i = i f_1 \]
and it is called the $i$-th harmonic wave.

\cim{images/ch18/fig18_13a}{0.4}
