\chapter{Image Formation}

When light encounters a surface, images can be formed through reflection or refraction.
This allows us to design mirros and lenses to form images that we want.

There are two main types of images.
\begin{itemize}
    \item \textbf{Real image}: image formed by real light rays intersecting at the image point,
        usually involves a lens
    \item \textbf{Virtual image}: image formed by extrapolating light rays to intersect at the 
        image point, usually involves a mirror
\end{itemize}

\section{Images Formed by Flat Mirrors}

Consider a point source of light placed at $O$ with distance $p$ from a flat mirror. $p$ is called 
the \textbf{object distance}. The light rays emitted from the source hits the mirror and gets
reflected. Extending the rays to \textit{behind} the mirror until they intersect at point $I$,
this forms a virtual image there with \textbf{image distance} $q$ from the mirror.
\cim{images/ch36/36_1.png}{0.5}
Note that while point source is just a point in space, we can combine many point sources together
to form an extended object.

Geometry can also be used to examine the properties of the image of extended objects. To do so,
pick a point of the object to be the point source, then pick two light rays originating from that point.
Suppose one of the rays originates from $P$ and moves perpendicular to the object and hits the mirror
at $Q$ and gets reflected straight back. The other ray follows path $PR$ and reflects as shown in
diagram below. For both reflected rays, we extend each toward \textit{behind} the wall until 
they intersect at point $P'$. This results in a virtual image.
\cim{images/ch36/36_2.png}{0.5}
Notice that $PQR$ and $P'QR$ are congruent, so $PQ = P'Q$ and $|p| = |q|$.

Suppose that our object has height $h$ and its image has height $h'$, the \textbf{lateral
magnification} $M$ can be defined as
\begin{equation}\label{36.1}
    M = \frac{h'}{h}
\end{equation}
This definition also works for any type of mirror and lenses. For a flat mirror, $M = 1$ in all cases
since $h' = h$. This also means that our image is up right because $M$ is positive.

Flat mirrors also make it seem like our image is flipped vertically. However, this is not actually
a left-right reversal but actually front-back.

\section{Images Formed by Spherical Mirrors}

\subsection{Concave Mirrors}

\cim{images/ch36/36_6.png}{0.7}

Consider a concave mirror with radius $r$ and center of curvature located at point $C$. This type
of mirror focuses parallel incoming light rays to a real focus point $F$. The distance $f$ from $F$ to
a perpendicular point of the mirror can be computed by $f = r/2$.

Now, consider a point source of light placed at $O$, lying on the central axis line somewhere
\textit{behind} $C$ and two light rays diverging from it. Note that the two light rays being considered
make small angles with the central axis, and are called \textbf{paraxial rays}. If the angle is too big,
the rays will converge to a point that is not $F$ and creates a blurred image. This effect is called 
\textbf{spherical aberration}.

If the distance $p$ and radius $r$ are already known, we can compute the image distance $q$ from them.
Consider two rays of light coming from the same point on object $O$. One ray perpendicularly hit the 
mirror and reflects right back while the other touches the mirror where the central axis touches it.
We can then use the law of reflection to find where this ray will reflect. The point $I$ at which 
the reflected rays intersect will be the location of where the identical part of the image will be.
\cim{images/ch36/36_9.png}{0.75}
Notice that $\tan\theta = h/p$ (orange triangle) and $\tan\theta = -h'q$ (yellow triangle) where the 
negative sign is there because the image is inverted. Thus, we can find the magnification $M$ by 
\begin{equation}\label{36.2}
    M = \frac{h'}{h} = - \frac{p}{q}
\end{equation}

\begin{itemize}
    \item $q$ is positive for real image and negative for virtual image 
    \item $p$ is positive for real object and negative for virtual object 
    \item $m>0$ when image is upright 
    \item $m<0$ when image is inverted
\end{itemize}

Furthermore, we can follow this table for sign conventions for mirrors.
\cim{images/ch36/t36_1.png}{1}

\subsection{Convex Mirror}

There is no need to derive any equation because we can use the same equations as with concave mirrors.
The biggest difference is that the image will now be virtual and always upright.

\subsection{Mirror Equation}

\begin{equation}
    \frac{1}{p} + \frac{1}{q} = \frac{1}{f}
\end{equation}
where $f$ is the focal length.

\section{Images Formed by Thin Lenses}

Lens is an optical system with two refracting surfaces. To simplify things, we assume that two 
spherical surfaces of the lens are close enough we can just ignore the distance between them.

Here, focal length $f$ is the distance from the lens to the point where the parallel rays converge,
both really and virtually.
\cim{images/ch36/36_23a.png}{0.6}

The position of the image formed by a thin lens can be computed using the \textbf{thin lens equation}
which looks very similar to the mirror equation.
\begin{equation}
    \frac{1}{p} + \frac{1}{q} = \frac{1}{f}
\end{equation}
\begin{equation}
    M = \frac{h'}{h} = - \frac{p}{q}
\end{equation}
However, notice that the real/virtual-ness of the images are opposite of mirrors. Thus, we can use 
the following sign conventions for thin lenses instead.
\cim{images/ch36/t36_3.png}{1}

\subsection{Combinations of Thin Lenses}

When combining thin lenses, the image formed by the previous lens is the object for the next lens.
Here, the total magnification is computed as a product of every magnification of every lens.
\[ m_{total} = m_1 m_2 m_3 \cdots m_n \]