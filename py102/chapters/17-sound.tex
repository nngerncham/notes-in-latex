\chapter{Sound Waves}

\textbf{Sound waves} are mechanical waves that can travel through any material, though 
most commonly through air. In this case, the disturbed elements are the air particles.
The disturbance leads to changes in the pressure and density of the air.

There are three categories of sound waves based on their frequency range:
\begin{itemize}
    \item \textbf{Audible waves}: frequency within range that human can hear
    \item \textbf{Infrasonic waves}: frequency below audible range
    \item \textbf{Ultrasonic waves}: frequency above audible range
\end{itemize}

\section{Pressure Variations in Sound Waves}

To describe sound waves, picture a 1D longitudinal pulse moving through a tube filled with 
compressible gas. There also exists a piston on the left that can move to compress the gas to 
create the pulse. When the piston extends, it compresses the gas which changes the pressure and 
density of the gas in the region right next to the piston. Once the piston stops, the compressed 
region of gas moves in the direction of the push of the piston through the tube with speed $v$,
shown on the left diagram. Additionally, the piston can move in a simple harmonic motion
to create a periodic wave in the tube as shown in the right diagram.
\cim{images/waves-and-sound/fig17_1.png}{0.7}

Notice that once the piston moves into the tube, it creates a compressed region $-$ \textbf{compression}
$-$ which moves through the tube and continuously compresses the gas right in front of it. Also note 
that the pressure and density of the gas in this region is higher than equilibrium.
Once the piston moves back, the gas in front of the piston expands and the pressure and density then 
becomes lower than equilibrium. This region is called \textbf{rarefactions}. Both regions will
propogate along the tube as shown in the diagram above.

Since the sound wave is longitudinal, any small element of the gas also move with simple harmonic
motion parallel to the direction of the wave. Let $s(x, t)$ be the position of the small element 
relative to its equilibrium position, we can use it to describe the harmonic position function as 
\begin{equation}\label{17.1}
    s(x, t) = s_{\max}\cos(kx - \omega t)
\end{equation}
where $s_{\max}$ is the \textbf{displacement amplitude}, describing the maximum displacement of the
small element from equilibrium. Like other waves, $k$ represents the wave number and $\omega$
represents the angular frequency. Note that the displacement is horizontal, opposed to vertical 
with the waves from the previous chapter.

\section{Intensity of Periodic Sound Waves}

Consider a sound source emitting sound wave in a sphere with radius $r$ centered at said source.
This creates a \textbf{spherical wave}, assuming that the air around the source is perfectly uniform.
The average power $P_{avg}$ must be distributed uniformly over this sphere. Here, we can define 
the \textbf{intensity} $I$ of a wave as the rate of energy transported by the wave per unit area of
the sphere. Recall that the area of a sphere with radius $R$ is computed by $A = 4\pi R^2$,
we can compute the intensity at distance $r$ from source with 
\begin{equation}\label{17.13}
    I = \frac{P_{avg}}{A} = \frac{P_{avg}}{4\pi r^2}
\end{equation}
Notice that this decreases the furthur you are from the source.

\subsection{Sound Level in Decibels}

It is more convenient to represent the different intensities the human ear can hear with a
logarithmic scale since the range is so wide. Thus, we can define the \textbf{sound level} $\beta$
as \begin{equation}\label{17.14}
    \beta = 10\log\left(\frac{I}{I_0}\right)
\end{equation}
Constant $I_0$ is known as the \textbf{reference intensity}, describing the threshold of hearing and
has the value of $10^{-12} W/m^2$. $I$ is the intensity in watts per square meter at which $\beta$
corresponds to. The unit for this is \textbf{decibels} (dB).

The following table shows the sound level of different objects
\cim{images/waves-and-sound/tab17_2.png}{0.3}

\section{Doppler Effect}

The \textbf{Doppler effect} describes the phenomena where the frequency of the sound you just heard
is higher the closer the source is to you. For a high-level explanation, suppose that you are an
observer of sound waves coming from an object that is the source of sound. If you are at rest,
the number of waves reaching you per fixed time will be constant. However, if you move towards the 
object, more waves will reach you in the same unit time since the distance between you and the object 
decreases, causing the frequency to change.

Now, we can describe the effect using the following equation. Additionally, we define the following
parameters
\begin{itemize}
    \item $f_S$: frequency at the source
    \item $f_O$: frequency observed
    \item $\lambda$: source wavelength 
    \item $v_S$: speed of source
    \item $v_O$: speed of observer
\end{itemize}

\subsection{Source at Rest, Observer in Motion}

First, recall that $v = \lambda/T$, we can express frequency as
\[ f = \frac{v}{\lambda}, \lambda = \frac{v}{f} \]
Since we are moving with a moving object as part of our model, we replace
$v$ with relative velocity. If the two objects are moving towards each other, we use
\begin{equation}\label{17.15}
    f_O = \frac{v_{SO}}{\lambda} = \frac{v - (-v_O)}{\lambda} = \frac{v + v_O}{\lambda}
        = \left(\frac{v + v_O}{v}\right)f_S
\end{equation}

If the observer is moving away from the source, simply change $v_{SO} = v_S - v_O$. The reason
for this is because when $O$ is moving towards $S$, the change in distance between the two is
negative. In contrast, if $O$ is moving away from $S$, the change in distance between the two is 
positive.

\subsection{Source in Motion, Observer at Rest}

Let us consider the case where the source is moving tward the observer.
Similar to when the source is at rest and observer is in motion, the frequency increases and thus 
the wavelength decreases. However, let us now focus on the wavelength instead. For every wave emitted
$-$ each lasting a time interval $T$ (period) $-$, the source moves by a distance of $v_S T = v_s/f$.
Thus, the wavelength is shortened by that amount. Therefore, the observed wavelength will become 
\begin{equation}
\lambda_O = \lambda - v_S T = \lambda - \frac{v_S}{f_S}
\end{equation}
Recall that $f = v/\lambda$ and $v = \lambda f$, we can plug the observed wavelength in to obtain 
\begin{equation}\label{17.17}
f_O = \frac{v}{\lambda_O} = \frac{v f_S}{\lambda f_S - v_S}
    = \left(\frac{v}{v - v_S}\right)f_S
\end{equation}

If the source is moving away from the observer, simply flip the sign for $v_S$ in $v_O - v_S$.

\subsection{General Case}

To generalize the Doppler effect for two possibly moving objects source and observer, we can combine
equations~\eqref{17.15} and~\eqref{17.17} to obtain 
\begin{equation}\label{17.19}
    f_O = \left(\frac{v + v_O}{v - v_S}\right)f_S
\end{equation}
where the signs depend on if each object is moving and in what direction. The positive represents
when an object is moving towards the other, while negative represents moving away.